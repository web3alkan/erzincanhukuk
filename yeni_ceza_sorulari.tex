\documentclass[12pt, a4paper]{article}
\usepackage[turkish]{babel}
\usepackage[utf8]{inputenc}
\usepackage[T1]{fontenc}
\usepackage{geometry}
\usepackage{multicol}
\usepackage{enumitem}
\usepackage{amssymb}
\usepackage{amsmath}
\usepackage{amsfonts}
\usepackage{graphicx}
\usepackage{fancyhdr}
\usepackage{lipsum} % For dummy text

% Page geometry
\geometry{
    a4paper,
    left=20mm,
    right=20mm,
    top=25mm,
    bottom=25mm
}

% Header and Footer
\pagestyle{fancy}
\fancyhf{}
\fancyhead[C]{Ceza Hukuku Özel Hükümler - Kapsamlı Soru Bankası}
\fancyfoot[C]{\thepage}

% Section styling
\usepackage{titlesec}
\titleformat{\section}
  {\normalfont\Large\bfseries\scshape}
  {}
  {0em}
  {}[\titlerule]

\newcommand{\topic}[1]{\par\vspace{4mm}\noindent\large\textbf{#1}\par\vspace{1mm}\nopagebreak}
\newcommand{\question}[1]{\subsection*{#1}}
\newcommand{\options}[5]{
    \begin{enumerate}[label=\Alph*)]
        \item #1
        \item #2
        \item #3
        \item #4
        \item #5
    \end{enumerate}
}
\newcommand{\answer}[3]{
    \vspace{5mm}
    \noindent\textbf{Doğru Cevap:} #1
    \vspace{3mm}
    
    \noindent\textbf{Hukuki Açıklama:} #2
    \vspace{3mm}
    
    \noindent\textbf{İlgili Kanun Maddesi:}
    \begin{quotation}
        \small
        \textit{#3}
    \end{quotation}
    \vspace{5mm}
    \hrule
}

\begin{document}

\title{\textbf{Ceza Hukuku Özel Hükümler Kapsamlı Soru Bankası (Bölüm 1/4)}}
\author{Oluşturan: Yapay Zeka Destekli Asistan}
\date{\today}
\maketitle

\thispagestyle{empty}
\newpage

\section*{Ceza Hukuku Özel Hükümler Soruları (1-10. Konular)}

\topic{1. Göçmen Kaçakçılığı (Uluslararası Suçlar)}
\question{Soru 1: R, yasa dışı yollarla Avrupa'ya gitmek isteyen bir grup yabancı uyruklu kişiden para alarak, onları bir tekneyle Türkiye'den Yunanistan'a geçirmeyi organize etmiştir. R'nin eylemi hangi suçu oluşturur?}
\options
{İrtikap}
{Nitelikli dolandırıcılık}
{İnsan ticareti}
{Göçmen kaçakçılığı}
{Devletin birliğini ve ülke bütünlüğünü bozmak}
\answer{D) Göçmen kaçakçılığı}
{Göçmen kaçakçılığı suçu, maddi bir menfaat elde etmek amacıyla, yasal olmayan yollardan bir yabancının yurt dışına çıkmasına imkan sağlamaktır. R'nin eylemi TCK m. 79'da düzenlenen bu suçu oluşturur.}
{\textbf{TCK m. 79 - Göçmen Kaçakçılığı} \\ (1) Doğrudan doğruya veya dolaylı olarak maddi bir menfaat elde etmek maksadıyla, yasal olmayan yollardan; ... b) Türk vatandaşı veya yabancının yurt dışına çıkmasına imkan sağlayan, Kişi, beş yıldan sekiz yıla kadar hapis ve bin günden onbin güne kadar adlî para cezası ile cezalandırılır.}

\question{Soru 2: Bir otel işletmecisi olan A, yasa dışı yollarla ülkeye girmiş bir grup göçmenin, yurt dışına çıkmak için organizatör beklerken haftalarca otelinde kalmasına, bu durumlarını bilerek ve fahiş bir ücret karşılığında izin vermiştir. A'nın eylemi nedir?}
\options
{Suçluyu kayırma}
{Nitelikli dolandırıcılık}
{Göçmen kaçakçılığı}
{Yardım ve yataklık}
{Suç değildir, ticari faaliyettir}
\answer{C) Göçmen kaçakçılığı}
{Göçmen kaçakçılığı suçu, sadece bir yabancıyı ülkeye sokmak veya yurt dışına çıkarmakla değil, aynı zamanda maddi menfaat karşılığında, yasa dışı olarak ülkede kalmasına imkan sağlamakla da işlenir. Otel işletmecisi A'nın eylemi, TCK m. 79/1-a'daki bu seçimlik harekete uymaktadır.}
{\textbf{TCK m. 79 - Göçmen Kaçakçılığı} \\ (1) Doğrudan doğruya veya dolaylı olarak maddi bir menfaat elde etmek maksadıyla, yasal olmayan yollardan; a) Bir yabancıyı ülkeye sokan veya ülkede kalmasına imkan sağlayan... Kişi, beş yıldan sekiz yıla kadar hapis... cezası ile cezalandırılır.}

\question{Soru 3: İnsan kaçakçısı B, kapasitesi 20 kişi olan bir lastik bota 80 kişiyi bindirerek denize açılmış, ancak botun alabora olması sonucu birçok göçmen boğularak ölmüştür. Bu olayda B'nin ceza sorumluluğu nasıl belirlenir?}
\options
{Sadece göçmen kaçakçılığı suçundan cezalandırılır.}
{Göçmen kaçakçılığı ve olası kastla öldürme suçlarından ayrı ayrı cezalandırılır.}
{Suçun nitelikli halinden (insan hayatı tehlikeye atılması) ve ayrıca taksirle öldürmeden cezalandırılır.}
{Göçmen kaçakçılığı suçunun neticesi sebebiyle ağırlaşmış halinden cezalandırılır.}
{Olay bir kaza olduğu için sadece tazminat sorumluluğu doğar.}
\answer{D) Göçmen kaçakçılığı suçunun neticesi sebebiyle ağırlaşmış halinden cezalandırılır.}
{Göçmen kaçakçılığı suçunun, mağdurların hayatı bakımından bir tehlike oluşturması veya onur kırıcı bir muameleye maruz bırakılarak işlenmesi nitelikli haldir. Eğer bu fiiller neticesinde ölüm meydana gelirse, faile verilecek ceza TCK m. 80 uyarınca ağırlaştırılır. Bu, özel bir neticesi sebebiyle ağırlaşmış suç halidir.}
{\textbf{TCK m. 80 - İnsan Ticareti} \\ (1) Göçmen kaçakçılığı suçunun işlenmesi sırasında; ... c) Mağdurların hayatı bakımından bir tehlike oluşturması, d) Onur kırıcı bir muameleye maruz bırakılarak işlenmesi, halinde, verilecek ceza yarısından üçte ikisine kadar artırılır. (2) Bu suçun işlenmesi sırasında kasten yaralama suçunun neticesi sebebiyle ağırlaşmış hallerinin veya ölümün meydana gelmesi durumunda, ayrıca bu suçlardan dolayı da cezaya hükmolunur.}

\topic{2. Kasten Öldürme Suçu (Kişilere Karşı Suçlar)}
\question{Soru 4: A, aralarındaki bir alacak verecek meselesi yüzünden tartıştığı B'yi öldürmek amacıyla ruhsatsız tabancasıyla üç el ateş etmiş ve B'nin ölümüne neden olmuştur. A'nın eylemi hangi suçu oluşturur?}
\options
{Taksirle öldürme}
{Kasten yaralama sonucu ölüme neden olma}
{Kasten öldürme}
{Kan gütme saikiyle öldürme}
{Töre saikiyle öldürme}
\answer{C) Kasten öldürme}
{Kasten öldürme suçu, bir insanın hayatına bilerek ve isteyerek son vermektir. A'nın, B'yi öldürme kastıyla hareket ederek ateş etmesi ve bu eylem sonucunda B'nin ölmesi, TCK m. 81'de düzenlenen kasten öldürme suçunun temel halini oluşturur.}
{\textbf{TCK m. 81 - Kasten Öldürme} \\ (1) Bir insanı kasten öldüren kişi, müebbet hapis cezası ile cezalandırılır.}

\question{Soru 5: C, miras yüzünden anlaşmazlık yaşadığı babası D'yi öldürmeye bir hafta önceden karar verir. Bir plan yapar, zehir satın alır, uygun bir anı kollar ve D'nin kahvesine zehir katarak onun ölümüne neden olur. C'nin eylemi nedir?}
\options
{Basit kasten öldürme}
{Taksirle öldürme}
{Tasarlayarak ve üstsoya karşı kasten öldürme}
{Kan gütme saikiyle öldürme}
{Eziyet çektirerek öldürme}
\answer{C) Tasarlayarak ve üstsoya karşı kasten öldürme}
{Kasten öldürme suçunun tasarlayarak ve üstsoya (baba, anne vb.) karşı işlenmesi TCK m. 82'de nitelikli hal olarak düzenlenmiştir ve cezası ağırlaştırılmış müebbet hapistir. C'nin plan yapması "tasarlamayı", babasını öldürmesi ise "üstsoya karşı" işlemeyi gösterir.}
{\textbf{TCK m. 82 - Nitelikli Kasten Öldürme} \\ (1) Kasten öldürme suçunun; a) Tasarlayarak, ... d) Üstsoy veya altsoydan birine ... karşı, ... işlenmesi halinde, kişi ağırlaştırılmış müebbet hapis cezası ile cezalandırılır.}

\question{Soru 6: E'nin kardeşi, F'nin ailesinden biri tarafından yıllar önce öldürülmüştür. E, bu olayın intikamını almak amacıyla, olayla hiçbir ilgisi olmayan F'yi bir gün pusuya düşürerek öldürür. E'nin eylemi hangi suçu oluşturur?}
\options
{Kasten öldürme}
{Töre saikiyle öldürme}
{Kan gütme saikiyle kasten öldürme}
{Haksız tahrik altında kasten öldürme}
{Tasarlayarak öldürme}
\answer{C) Kan gütme saikiyle kasten öldürme}
{Bir başkasını, daha önce işlenmiş bir cinayetin yarattığı düşmanlık ve intikam duygusuyla (kan gütme saikiyle) öldürmek, TCK m. 82'de düzenlenen bir diğer nitelikli haldir. E'nin amacı, aileler arasındaki eski bir husumetin intikamını almak olduğu için bu nitelikli hal uygulanır.}
{\textbf{TCK m. 82 - Nitelikli Kasten Öldürme} \\ (1) Kasten öldürme suçunun; ... j) Kan gütme saikiyle, ... işlenmesi halinde, kişi ağırlaştırılmış müebbet hapis cezası ile cezalandırılır.}

\topic{3. İntihara Yönlendirme Suçu (Kişilere Karşı Suçlar)}
\question{Soru 7: G, ağır bir depresyon geçiren arkadaşı H'ye sürekli olarak hayatın anlamsız olduğunu, acılarından kurtulmasının tek yolunun ölmek olduğunu telkin etmiş, H'ye bir kutu ilacı "bunlarla acısız bir şekilde gidersin" diyerek vermiştir. H bu ilaçları içerek intihar etmiştir. G'nin eylemi nedir?}
\options
{Taksirle ölüme neden olma}
{Kasten öldürmenin ihmali davranışla işlenmesi}
{Yardım veya bildirim yükümlülüğünün yerine getirilmemesi}
{İntihara yönlendirme}
{Suçsuzdur, çünkü H kendi kararını vermiştir}
\answer{D) İntihara yönlendirme}
{İntihara yönlendirme suçu, başkasını intihara azmettirmek, teşvik etmek, başkasının intihar kararını kuvvetlendirmek ya da başkasının intiharına herhangi bir şekilde yardım etmek suretiyle işlenir. G'nin telkinleri ve araç temin etmesi bu suçu oluşturur.}
{\textbf{TCK m. 84 - İntihara Yönlendirme} \\ (1) Başkasını intihara azmettiren, teşvik eden, başkasının intihar kararını kuvvetlendiren ya da başkasının intiharına herhangi bir şekilde yardım eden kişi, iki yıldan beş yıla kadar hapis cezası ile cezalandırılır.}

\question{Soru 8: J, yönettiği bir internet sitesinde, gençleri intihara teşvik eden "oyunlar" düzenlemekte, onlara çeşitli görevler vererek ve "cesaretlerini" övgüyle karşılayarak intihar etmelerini sağlamaktadır. Bu site üzerinden birçok genç intihar etmiştir. J'nin eylemi nedir?}
\options
{Kasten öldürme}
{İntihara yönlendirme suçunun alenen işlenmesi}
{Halkı kin ve düşmanlığa tahrik}
{Suç işlemeye alenen teşvik}
{Sadece TİB tarafından sitenin kapatılması gerekir, suç oluşmaz}
\answer{B) İntihara yönlendirme suçunun alenen işlenmesi}
{İntihara yönlendirme suçunun basın ve yayın yolu ile (alenen) işlenmesi halinde ceza artırılır. J'nin internet sitesi üzerinden birden fazla kişiye açık bir şekilde bu eylemi gerçekleştirmesi, suçun nitelikli halini oluşturur.}
{\textbf{TCK m. 84 - İntihara Yönlendirme} \\ (3) İntihara teşvik fiilinin basın ve yayın yolu ile işlenmesi hâlinde, kişi dört yıldan on yıla kadar hapis cezası ile cezalandırılır.}

\question{Soru 9: K, ortağı L'nin iflas ettiğini ve intiharın eşiğinde olduğunu bilmektedir. L, K'ye "Sanırım bu işin sonu intihar olacak" dediğinde, K, şirketin sigorta parasını alabilmek için "Haklısın, başka çare kalmadı, en azından borçları kapatırız" diyerek L'nin zaten var olan intihar düşüncesini pekiştirir. L, bu konuşmadan sonra intihar eder. K'nin sorumluluğu nedir?}
\options
{Suçsuzdur, çünkü sadece fikrini söylemiştir}
{İntihara yönlendirme (kararı kuvvetlendirme)}
{Olası kastla öldürme}
{Taksirle ölüme neden olma}
{Dolandırıcılık}
\answer{B) İntihara yönlendirme (kararı kuvvetlendirme)}
{İntihara yönlendirme suçu, sadece birini intihara teşvik etmekle değil, aynı zamanda o kişinin zaten mevcut olan intihar kararını kuvvetlendirmek suretiyle de işlenebilir. K'nin, L'nin kararını onaylayarak onu bu yönde cesaretlendirmesi, bu seçimlik hareketi oluşturur.}
{\textbf{TCK m. 84 - İntihara Yönlendirme} \\ (1) Başkasını intihara ... kararını kuvvetlendiren ... kişi, iki yıldan beş yıla kadar hapis cezası ile cezalandırılır.}

\topic{4. Taksirle Öldürme Suçu (Kişilere Karşı Suçlar)}
\question{Soru 10: Avcı M, ormanlık alanda avlanırken, çalıların arkasından gelen bir hışırtı üzerine o yöne doğru, hedefin ne olduğunu tam olarak kontrol etmeden ateş eder. Ancak çalıların arkasında başka bir avcı olan N bulunmaktadır ve N vurularak hayatını kaybeder. M'nin eylemi nedir?}
\options
{Kasten öldürme}
{Olası kastla öldürme}
{Bilinçli taksirle öldürme}
{Basit taksirle öldürme}
{Kazaen adam öldürme, suç oluşmaz}
\answer{D) Basit taksirle öldürme}
{Taksir, dikkat ve özen yükümlülüğüne aykırılık dolayısıyla bir davranışın, suçun kanuni tanımında belirtilen neticesi öngörülmeyerek gerçekleştirilmesidir. M'nin, hedefi kontrol etme yükümlülüğünü ihlal etmesi ve öngörmediği bu ölüm neticesine sebep olması, basit taksirle öldürme suçunu oluşturur.}
{\textbf{TCK m. 85 - Taksirle Öldürme} \\ (1) Taksirle bir insanın ölümüne neden olan kişi, iki yıldan altı yıla kadar hapis cezası ile cezalandırılır.}

\question{Soru 11: Bir cerrah olan O, 36 saatlik uykusuz bir nöbetin ardından girdiği acil bir ameliyatta, aşırı yorgunluk nedeniyle dikkatini toplayamaz ve hastanın vücudunda yabancı bir cisim (tampon) unutur. Hasta, bu nedenle gelişen komplikasyonlar sonucu hayatını kaybeder. Cerrah O'nun ceza sorumluluğu nedir?}
\options
{Kasten öldürme}
{Basit taksirle öldürme}
{Bilinçli taksirle öldürme}
{Mesleki hata olduğu için suç oluşmaz}
{Olası kastla öldürme}
\answer{B) Basit taksirle öldürme}
{Herkesin, özellikle mesleği gereği, uyması gereken dikkat ve özen yükümlülükleri vardır. Bir cerrahın, yorgun da olsa, mesleğinin gerektirdiği temel standartlara uyması beklenir. Bu standarda uymayarak öngörülebilir bir neticeye (hastanın ölümü) neden olması, taksirli sorumluluğunu doğurur.}
{\textbf{TCK m. 22 - Taksir} \\ (2) Taksir, dikkat ve özen yükümlülüğüne aykırılık dolayısıyla, bir davranışın suçun kanuni tanımında belirtilen neticesi öngörülmeyerek gerçekleştirilmesidir.}

\question{Soru 12: P, kırmızı ışıkta "nasılsa araba gelmez" diyerek hızla geçtiği kavşakta, yeşil ışıkta geçmekte olan bir motosiklete çarparak sürücünün ölümüne neden olur. P, kırmızı ışıkta geçmenin tehlikeli olduğunu ve bir kazaya neden olabileceğini bilmekte, yani neticeyi öngörmekte ancak şansına güvenerek eylemini gerçekleştirmektedir. P'nin eylemi nedir?}
\options
{Olası kastla öldürme}
{Basit taksirle öldürme}
{Bilinçli taksirle öldürme}
{Kasten öldürme}
{Trafik güvenliğini tehlikeye sokma}
\answer{C) Bilinçli taksirle öldürme}
{Kişinin, neticeyi öngördüğü hâlde, fiili işlemesi durumunda bilinçli taksir vardır. Basit taksirden farkı, neticenin fail tarafından öngörülmüş olmasıdır. P, kaza ihtimalini öngörmesine rağmen "bir şey olmaz" diyerek hareket ettiği için bilinçli taksirden sorumludur ve cezası artırılır.}
{\textbf{TCK m. 22 - Taksir} \\ (3) Kişinin öngördüğü neticeyi istememesine karşın, neticenin meydana gelmesi halinde, bilinçli taksir vardır; bu halde taksirli suça ilişkin ceza üçte birinden yarısına kadar artırılır.}

\topic{5. Kasten Yaralama Suçu (Kişilere Karşı Suçlar)}
\question{Soru 13: Bir tartışma sırasında R, S'ye yumruk atarak burnunun kırılmasına neden olmuştur. Adli tıp raporunda, kırığın S'nin yaşamını tehlikeye sokmadığı ancak basit tıbbi müdahale ile giderilemeyeceği belirtilmiştir. R'nin eylemi hangi suçu oluşturur?}
\options
{Taksirle yaralama}
{Neticesi sebebiyle ağırlaşmış yaralama (kemik kırığı)}
{İşkence}
{Basit kasten yaralama}
{Neticesi sebebiyle ağırlaşmış yaralama (yaşamsal tehlike)}
\answer{B) Neticesi sebebiyle ağırlaşmış yaralama (kemik kırığı)}
{Kasten yaralama fiilinin vücutta kemik kırılmasına neden olması, TCK m. 87/3 uyarınca cezanın artırılmasını gerektiren bir neticesi sebebiyle ağırlaşmış haldir. R'nin eylemi bu suçu oluşturur.}
{\textbf{TCK m. 86 - Kasten Yaralama} \\ (1) Kasten başkasının vücuduna acı veren veya sağlığının ya da algılama yeteneğinin bozulmasına neden olan kişi, bir yıldan üç yıla kadar hapis cezası ile cezalandırılır. \\ \textbf{TCK m. 87 - Neticesi Sebebiyle Ağırlaşmış Yaralama}}

\question{Soru 14: T, kıskançlık nedeniyle tartıştığı eşi U'nun yüzüne kezzap (asit) atmıştır. Saldırı sonucunda U'nun yüzünde kalıcı bir iz ve çirkinleşme meydana gelmiştir. T'nin eylemi nedir?}
\options
{Basit kasten yaralama}
{İşkence}
{Eziyet}
{Kasten yaralamanın nitelikli hali (yüzde sabit iz)}
{Taksirle yaralama}
\answer{D) Kasten yaralamanın nitelikli hali (yüzde sabit iz)}
{Kasten yaralama fiili sonucunda mağdurun yüzünde sabit bir ize veya yüzünün sürekli değişikliğine neden olunması, suçun cezasını ağırlaştıran nitelikli bir haldir. T'nin eylemi, TCK m. 87/2-c bendi uyarınca nitelikli kasten yaralama suçunu oluşturur.}
{\textbf{TCK m. 87 - Neticesi Sebebiyle Ağırlaşmış Yaralama} \\ (2) Kasten yaralama fiili, mağdurun; ... c) Yüzünde sabit bir ize, d) Yüzünün sürekli değişikliğine, ... neden olmuşsa, yukarıdaki maddeye göre belirlenen ceza, iki kat artırılır.}

\question{Soru 15: V, bir kamu görevlisi olan ve görevini yapmakta olan zabıta memuru Y'ye, kendisini dükkanından çıkarmaya çalıştığı için direnmiş ve onu iterek düşürüp kolunu kırmasına neden olmuştur. V'nin eylemi nasıl nitelendirilir?}
\options
{Görevi yaptırmamak için direnme}
{Basit kasten yaralama}
{Kamu görevlisine karşı işlenen nitelikli kasten yaralama}
{Cebir}
{Memura hakaret}
\answer{C) Kamu görevlisine karşı işlenen nitelikli kasten yaralama}
{Kasten yaralama suçunun, kişinin yerine getirdiği kamu görevi nedeniyle işlenmesi, TCK m. 86/3-c bendi uyarınca suçun nitelikli halidir ve şikayete tabi değildir, cezası artırılır. V'nin eylemi hem kamu görevlisine karşı hem de kemik kırığına neden olduğu için birden fazla nitelikli hali barındırır.}
{\textbf{TCK m. 86 - Kasten Yaralama} \\ (3) Kasten yaralama suçunun; ... c) Kişinin yerine getirdiği kamu görevi nedeniyle, ... işlenmesi halinde, şikâyet aranmaksızın, verilecek ceza yarı oranında artırılır.}

\topic{6. Taksirle Yaralama Suçu (Kişilere Karşı Suçlar)}
\question{Soru 16: Sürücü Z, yerleşim yeri içinde hız sınırını aşarak ve dikkatsiz bir şekilde araç kullanırken, yaya geçidinden geçmekte olan A'ya çarpmış ve A'nın ağır yaralanmasına (basit tıbbi müdahale ile giderilemeyecek şekilde) neden olmuştur. Z'nin amacı A'ya çarpmak değildir. Z'nin eylemi nedir?}
\options
{Kasten yaralama}
{Taksirle yaralama}
{Trafik güvenliğini tehlikeye sokma}
{Bilinçli taksirle yaralama}
{Olası kastla yaralama}
\answer{B) Taksirle yaralama}
{Taksir, dikkat ve özen yükümlülüğüne aykırılık dolayısıyla, bir davranışın suçun kanuni tanımında belirtilen neticesi öngörülmeyerek gerçekleştirilmesidir. Z'nin eylemi, TCK m. 89'daki taksirle yaralama suçunu oluşturur.}
{\textbf{TCK m. 89 - Taksirle Yaralama} \\ (1) Taksirle başkasının vücuduna acı veren veya sağlığının ya da algılama yeteneğinin bozulmasına neden olan kişi, üç aydan bir yıla kadar hapis veya adlî para cezası ile cezalandırılır.}

\question{Soru 17: Bir inşaat sahibi olan B, şantiyede gerekli güvenlik önlemlerini (baret takma zorunluluğu, güvenlik ağı vb.) almamıştır. Bu ihmal sonucu, üst kattan düşen bir tuğla, aşağıda çalışan işçi C'nin başına isabet ederek onu ağır şekilde yaralamıştır. B'nin sorumluluğu nedir?}
\options
{Kasten yaralama}
{Olası kastla yaralama}
{Taksirle yaralama}
{İş kazasıdır, cezai sorumluluk yoktur}
{Eziyet}
\answer{C) Taksirle yaralama}
{İşverenlerin, iş sağlığı ve güvenliği konusunda gerekli önlemleri alma yükümlülüğü vardır. Bu yükümlülüğe aykırı davranarak bir yaralanmaya neden olmak, dikkat ve özen yükümlülüğünün ihlali anlamına gelir ve taksirli sorumluluğunu doğurur. B, gerekli önlemleri almadığı için taksirle yaralamadan sorumludur.}
{\textbf{TCK m. 89 - Taksirle Yaralama} \\ (1) Taksirle başkasının vücuduna acı veren veya sağlığının ya da algılama yeteneğinin bozulmasına neden olan kişi, üç aydan bir yıla kadar hapis veya adlî para cezası ile cezalandırılır.}

\question{Soru 18: D, evinin balkonunda bulunan büyük ve ağır bir saksıyı, devrilmeyecek şekilde sabitlememiştir. Şiddetli bir rüzgar sırasında saksı balkondan düşmüş ve o sırada yoldan geçmekte olan E'nin omzuna çarparak kemik kırığına neden olmuştur. D'nin eylemi nedir?}
\options
{Kasten yaralama}
{Bilinçli taksirle yaralama}
{Olası kastla yaralama}
{Basit taksirle yaralama}
{Genel güvenliği tehlikeye sokma}
\answer{D) Basit taksirle yaralama}
{Kişilerin, eylemlerinin ve ihmallerinin öngörülebilir sonuçlarına karşı dikkatli ve özenli olma yükümlülüğü vardır. D'nin saksıyı sabitlememesi bu yükümlülüğün ihlalidir ve öngörülebilir bir netice olan yaralanmaya yol açtığı için taksirli sorumluluğunu gündeme getirir.}
{\textbf{TCK m. 22 - Taksir} \\ (2) Taksir, dikkat ve özen yükümlülüğüne aykırılık dolayısıyla, bir davranışın suçun kanuni tanımında belirtilen neticesi öngörülmeyerek gerçekleştirilmesidir.}

\topic{7. İşkence Suçu (Kişilere Karşı Suçlar)}
\question{Soru 19: Bir polis memuru olan F, bir soruşturma kapsamında gözaltına aldığı şüpheli G'ye, suçunu itiraf etmesi için sistematik olarak birkaç gün boyunca fiziksel şiddet uygulamış, uykusuz bırakmış ve aşağılayıcı muamelelerde bulunmuştur. F'nin eylemi hangi suçu oluşturur?}
\options
{Kasten yaralama}
{Görevi kötüye kullanma}
{Eziyet}
{İşkence}
{Cebir}
\answer{D) İşkence}
{İşkence suçu, bir kişiye karşı insan onuruyla bağdaşmayan ve bedensel veya ruhsal yönden acı çekmesine, algılama veya irade yeteneğinin etkilenmesine, aşağılanmasına yol açacak davranışların bir kamu görevlisi tarafından sistematik olarak gerçekleştirilmesidir. F'nin eylemleri bu suçu oluşturur.}
{\textbf{TCK m. 94 - İşkence} \\ (1) Bir kişiye karşı insan onuruyla bağdaşmayan ve bedensel veya ruhsal yönden acı çekmesine, algılama veya irade yeteneğinin etkilenmesine, aşağılanmasına yol açacak davranışları gerçekleştiren kamu görevlisi hakkında sekiz yıldan onbeş yıla kadar hapis cezasına hükmolunur.}

\question{Soru 20: H, avukatı olan J'nin huzurunda ifade verirken, sorguyu yürüten komiser K, H'nin konuşmasını beğenmeyerek ona tokat atar ve "Doğru konuş, yoksa bu odadan çıkamazsın" der. Bu eylem tek seferliktir. Komiser K'nin eylemi nedir?}
\options
{İşkence}
{Eziyet}
{Tehdit ve kasten yaralama}
{Görevi kötüye kullanma}
{Cebir}
\answer{C) Tehdit ve kasten yaralama}
{İşkence suçunun oluşması için eylemlerin sistematik olması ve belli bir ağırlığa ulaşması gerekir. Tek bir tokat ve tehdit, işkence suçunu oluşturmayabilir. Bu durumda, tokat atma eylemi kasten yaralama (kamu görevlisine karşı işlenmesi nitelikli haldir), "çıkamazsın" sözü ise tehdit suçunu oluşturur ve gerçek içtima kuralları uygulanır.}
{\textbf{TCK m. 86 - Kasten Yaralama} ve \textbf{TCK m. 106 - Tehdit}}

\question{Soru 21: Bir cezaevinde görevli gardiyan L, sevmediği mahkum M'ye, onu aşağılamak ve acı çektirmek amacıyla, bir ay boyunca her gün soğuk suyla ıslatmış, yemeğine tükürmüş ve uykusunda sürekli rahatsız etmiştir. Bu eylemler sonucunda M ruhsal bir çöküntü yaşamıştır. L'nin eylemi nedir?}
\options
{Hakaret}
{Eziyet}
{İşkence}
{Görevi Kötüye Kullanma}
{Kasten Yaralama}
\answer{C) İşkence}
{İşkence, sadece fiziksel acıyı değil, ruhsal yönden acı çekmeye veya aşağılanmaya yol açan davranışları da kapsar. Gardiyan L'nin eylemlerinin bir ay boyunca sistematik olarak devam etmesi ve M'nin ruhsal sağlığını bozacak ve insan onurunu rencide edecek nitelikte olması, işkence suçunun unsurlarını oluşturur.}
{\textbf{TCK m. 94 - İşkence} \\ (1) Bir kişiye karşı insan onuruyla bağdaşmayan ve bedensel veya ruhsal yönden acı çekmesine, algılama veya irade yeteneğinin etkilenmesine, aşağılanmasına yol açacak davranışları gerçekleştiren kamu görevlisi ...}

\topic{8. Eziyet Suçu (Kişilere Karşı Suçlar)}
\question{Soru 22: N, birlikte yaşadığı ve kendisinden ayrılmak isteyen O'ya, sistematik olmamakla birlikte, sırf acı çektirmekten zevk aldığı için belirli aralıklarla tokat atmakta, üzerine soğuk su dökmekte ve küçük düşürücü sözler söylemektedir. N'nin amacı bilgi almak veya belli bir eyleme zorlamak değildir. N'nin suçu nedir?}
\options
{İşkence}
{Eziyet}
{Kasten yaralama}
{Tehdit}
{Huzur ve sükunu bozma}
\answer{B) Eziyet}
{Eziyet suçu, bir kimseye karşı acı verme kastıyla hareket edilen kötü muamele fiillerini kapsar. Olayda N'nin amacı sistematik bir amaca yönelik değil, sırf eziyet etmektir. Bu nedenle eylemi TCK m. 96'daki eziyet suçunu oluşturur.}
{\textbf{TCK m. 96 - Eziyet} \\ (1) Bir kimsenin eziyet çekmesine yol açacak davranışları gerçekleştiren kişi hakkında iki yıldan beş yıla kadar hapis cezasına hükmolunur.}

\question{Soru 23: P, kendisinden borç para isteyen ve geri ödemeyen arkadaşı R'yi, "terbiye etmek" amacıyla bir hafta boyunca her gün evine giderek ona hakaret etmiş, küçük düşürücü şakalar yapmış ve fiziksel olarak hafifçe itip kakmıştır. P'nin eylemleri R üzerinde ciddi bir manevi baskı oluşturmuştur. P'nin eylemi nedir?}
\options
{Hakaret ve Kasten Yaralama}
{İşkence}
{Şantaj}
{Eziyet}
{Cebir}
\answer{D) Eziyet}
{Eziyet suçu, işkence suçunun ağırlığına ulaşmayan, ancak süreklilik gösteren ve mağdura acı ve ıstırap çektirme kastı taşıyan eylemleri kapsar. P'nin bir hafta boyunca devam eden ve R'yi yıldırmaya yönelik aşağılayıcı ve saldırgan eylemleri, tek tek ele alındığında farklı suçlar oluştursa da, bir bütün olarak eziyet suçunun tanımına daha uygun düşmektedir.}
{\textbf{TCK m. 96 - Eziyet} \\ (1) Bir kimsenin eziyet çekmesine yol açacak davranışları gerçekleştiren kişi hakkında iki yıldan beş yıla kadar hapis cezasına hükmolunur.}

\question{Soru 24: S, evinde beslediği köpeğe sürekli olarak vurmakta, onu aç bırakmakta ve bu şekilde hayvana acı çektirmekten zevk almaktadır. S'nin bu eylemi TCK açısından bir suç oluşturur mu?}
\options
{Evet, eziyet suçunu oluşturur.}
{Evet, mala zarar verme suçunu oluşturur.}
{Hayvanları Koruma Kanunu'na göre idari para cezası gerektirir, TCK'da suç değildir.}
{TCK'da hayvanlara karşı eziyet özel olarak düzenlenmemiştir.}
{Genel güvenliği tehlikeye sokma suçunu oluşturur.}
\answer{C) Hayvanları Koruma Kanunu'na göre idari para cezası gerektirir, TCK'da suç değildir.}
{Türk Ceza Kanunu'nda (TCK) düzenlenen eziyet suçu (m. 96), insanlara karşı işlenen fiilleri kapsar. Hayvanlara karşı kötü muamele ve eziyet, 5199 sayılı Hayvanları Koruma Kanunu kapsamında düzenlenmiş olup, bu kanuna göre cezai yaptırımı genellikle idari para cezasıdır. (Not: Yasal düzenlemeler değişebilmektedir, bu cevap mevcut genel duruma göredir).}
{\textbf{5199 s. Hayvanları Koruma Kanunu} - İlgili maddeler hayvanlara kötü muamelenin idari yaptırıma tabi olduğunu düzenler. TCK m. 96 ise sadece insanlara karşı işlenebilir.}

\topic{9. Cinsel Saldırı Suçu (Cinsel Dokunulmazlığa Karşı Suçlar)}
\question{Soru 25: T, sokakta yalnız yürüyen U'yu zorla bir ara sokağa çekmiş ve tehdit ederek U'nun vücuduna cinsel organını veya sair bir cismi sokmuştur. T'nin eylemi hangi suça vücut verir?}
\options
{Cinsel taciz}
{Kişiyi hürriyetinden yoksun kılma}
{Nitelikli cinsel saldırı}
{Basit cinsel saldırı}
{Reşit olmayanla cinsel ilişki}
\answer{C) Nitelikli cinsel saldırı}
{Cinsel davranışlarla bir kimsenin vücut dokunulmazlığını ihlâl etmek cinsel saldırı suçunu oluşturur. Fiilin vücuda organ veya sair bir cisim sokulması suretiyle gerçekleştirilmesi, suçun nitelikli halidir (TCK m. 102/2). T'nin eylemi bu tanıma uymaktadır.}
{\textbf{TCK m. 102 - Cinsel Saldırı} \\ (2) Fiilin vücuda organ veya sair bir cisim sokulması suretiyle gerçekleştirilmesi durumunda, oniki yıldan az olmamak üzere hapis cezasına hükmolunur.}

\question{Soru 26: V, bir partide tanıştığı Y'nin içkisine, onun haberi olmadan uyuşturucu bir madde katar. Y'nin bilincini kaybetmesinden faydalanarak onunla cinsel ilişkiye girer. Y'nin eylemi nedir?}
\options
{Kasten yaralama ve cinsel saldırı}
{Reşit olmayanla cinsel ilişki}
{Nitelikli cinsel saldırı (beden veya ruh bakımından kendini savunamayacak kişiye karşı)}
{Cinsel taciz}
{İlaç kötüye kullanımı}
\answer{C) Nitelikli cinsel saldırı (beden veya ruh bakımından kendini savunamayacak kişiye karşı)}
{Cinsel saldırı suçunun, beden veya ruh bakımından kendisini savunamayacak durumda bulunan kişiye karşı işlenmesi, suçun cezasını ağırlaştıran bir nitelikli haldir. V, Y'yi ilaçla kendini savunamayacak hale getirip eylemini gerçekleştirdiği için TCK m. 102/3-a uyarınca nitelikli cinsel saldırıdan sorumlu olur.}
{\textbf{TCK m. 102 - Cinsel Saldırı} \\ (3) Suçun; a) Beden veya ruh bakımından kendisini savunamayacak durumda bulunan kişiye karşı, ... işlenmesi halinde, yukarıdaki fıkralara göre verilecek ceza yarı oranında artırılır.}

\question{Soru 27: Bir jinekolog olan Z, muayene ettiği hastası A'ya, tıbbi bir gereklilik olmamasına rağmen, kendi cinsel arzusunu tatmin amacıyla cinsel organına dokunmuştur. Z'nin eylemi nasıl nitelendirilir?}
\options
{Basit cinsel saldırı}
{Cinsel taciz}
{Görevi kötüye kullanma}
{Kamu görevinin sağladığı nüfuz kötüye kullanılmak suretiyle nitelikli cinsel saldırı}
{Hastanın özel hayatının gizliliğini ihlal}
\answer{D) Kamu görevinin sağladığı nüfuz kötüye kullanılmak suretiyle nitelikli cinsel saldırı}
{Cinsel saldırı suçunun, kamu görevinin veya hizmet ilişkisinin sağladığı nüfuz kötüye kullanılmak suretiyle işlenmesi, TCK m. 102/3-c uyarınca nitelikli hal sayılır. Doktor Z, mesleğinin sağladığı güven ve nüfuzu kötüye kullandığı için daha ağır bir ceza ile cezalandırılacaktır.}
{\textbf{TCK m. 102 - Cinsel Saldırı} \\ (3) Suçun; ... c) Kamu görevinin, vesayet veya hizmet ilişkisinin sağladığı nüfuz kötüye kullanılmak suretiyle, ... işlenmesi hâlinde, yukarıdaki fıkralara göre verilecek ceza yarı oranında artırılır.}

\topic{10. Çocukların Cinsel İstismarı Suçu (Cinsel Dokunulmazlığa Karşı Suçlar)}
\question{Soru 28: 40 yaşındaki B, 14 yaşındaki komşusunun çocuğu C'yi, kandırarak evine çağırmış ve burada çocuğa yönelik cinsel içerikli eylemlerde bulunmuştur. Çocuğun yaşı ve kandırılmış olması nedeniyle rızası hukuken geçerli değildir. B'nin eylemi nedir?}
\options
{Cinsel saldırı}
{Cinsel taciz}
{Çocukların cinsel istismarı}
{Reşit olmayanla cinsel ilişki}
{Eziyet}
\answer{C) Çocukların cinsel istismarı}
{On beş yaşını tamamlamamış çocuklara karşı gerçekleştirilen her türlü cinsel davranış, cinsel istismar suçunu oluşturur. C 14 yaşında olduğu için, rızası olsa dahi B'nin eylemi TCK m. 103 uyarınca cinsel istismar suçudur.}
{\textbf{TCK m. 103 - Çocukların Cinsel İstismarı} \\ (1) Çocuğu cinsel yönden istismar eden kişi, sekiz yıldan on beş yıla kadar hapis cezası ile cezalandırılır. Cinsel istismarın sarkıntılık düzeyinde kalması hâlinde üç yıldan sekiz yıla kadar hapis cezasına hükmolunur.}

\question{Soru 29: D, sosyal medya üzerinden tanıştığı ve 13 yaşında olduğunu bildiği E'ye, sürekli olarak cinsel içerikli mesajlar ve müstehcen fotoğraflar göndermiştir. D ve E hiç fiziksel olarak bir araya gelmemiştir. D'nin eylemi nedir?}
\options
{Cinsel taciz}
{Elektronik ortamda cinsel istismar (sarkıntılık düzeyinde)}
{Müstehcenlik suçu}
{Hakaret}
{Suç değildir, çünkü fiziksel temas yoktur}
\answer{B) Elektronik ortamda cinsel istismar (sarkıntılık düzeyinde)}
{Çocuğa karşı işlenen ve ani, kesik hareketlerle gerçekleşen cinsel davranışlar "sarkıntılık" olarak nitelendirilir ve cinsel istismarın daha az cezayı gerektiren bir halidir. Bu eylemin elektronik haberleşme araçlarıyla işlenmesi mümkündür. D'nin eylemi, çocuğa yönelik sarkıntılık düzeyinde cinsel istismar suçunu oluşturur.}
{\textbf{TCK m. 103 - Çocukların Cinsel İstismarı} \\ (1) ... Cinsel istismarın sarkıntılık düzeyinde kalması hâlinde üç yıldan sekiz yıla kadar hapis cezasına hükmolunur.}

\question{Soru 30: Okul müdürü F, rehberlik yapma bahanesiyle odasına çağırdığı 16 yaşındaki öğrenci G'ye, cinsel içerikli imalarda bulunmuş ve öğrencinin omzuna ve saçlarına dokunarak onu okşamıştır. G bu durumdan rahatsız olmuş ve durumu ailesine anlatmıştır. F'nin eylemi nasıl nitelendirilir?}
\options
{Basit cinsel taciz}
{Eğitici tarafından işlenen nitelikli cinsel istismar}
{Reşit olmayanla cinsel ilişkiye teşebbüs}
{Görevi kötüye kullanma}
{Cinsel saldırı}
\answer{B) Eğitici tarafından işlenen nitelikli cinsel istismar}
{Cinsel istismar suçunun; eğitici, öğretici, bakıcı gibi çocuğun korunmasından sorumlu kişiler tarafından işlenmesi, TCK m. 103/3-c uyarınca nitelikli haldir ve cezayı ağırlaştırır. Okul müdürü F, hizmet ilişkisinin sağladığı nüfuzu kötüye kullanarak bu suçu işlemiştir.}
{\textbf{TCK m. 103 - Çocukların Cinsel İstismarı} \\ (3) Suçun; ... c) Hizmet ilişkisinin sağladığı nüfuz kötüye kullanılmak suretiyle, ... işlenmesi hâlinde, yukarıdaki fıkralara göre verilecek ceza yarı oranında artırılır.}

\end{document} 