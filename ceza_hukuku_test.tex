\documentclass[a4paper, 11pt, twocolumn]{article}
\usepackage[utf8]{inputenc}
\usepackage[turkish]{babel}
\usepackage{geometry}
\usepackage{enumitem}
\usepackage{amsfonts}
\usepackage{amsmath}
\usepackage{amssymb}
\usepackage{graphicx}
\usepackage{xcolor}
\usepackage{framed}
\usepackage{multicol}
\usepackage{lipsum}
\usepackage{tcolorbox}

\geometry{a4paper, left=1.5cm, right=1.5cm, top=2cm, bottom=2cm}
\setlength{\columnsep}{0.7cm}

\definecolor{boxcolor}{rgb}{0.9, 0.9, 0.95}
\definecolor{lawbg}{rgb}{0.95, 0.95, 0.98}

\newenvironment{lawbox}{%
    \par\noindent
    \begin{tcolorbox}[
        colback=lawbg,
        colframe=boxcolor,
        fonttitle=\bfseries,
        boxrule=0.5pt,
        left=3pt,
        right=3pt,
        top=3pt,
        bottom=3pt,
        width=\columnwidth,
        arc=2pt,
        before skip=0.2cm,
        after skip=0.2cm
    ]
    \footnotesize\itshape
}{\end{tcolorbox}}

\title{
    \Huge \textbf{Ceza Hukuku Özel Hükümler} \\
    \large \textbf{Genişletilmiş Pratik Çalışma Test Sınavı}
}
\author{Ankara Hacı Bayram Veli Üniversitesi Hukuk Fakültesi \\ Yıllık Sınav Sorularından Uyarlanmıştır}
\date{Ocak 2024}

\begin{document}

\twocolumn[\maketitle]

\section*{SORULAR VE AÇIKLAMALI CEVAPLAR}
\vspace{0.3cm}

\subsection*{Soru 1}
Eşinden boşanan ve ailesiyle yaşayan S, iş yerindeki ilişkisinden hamile kalır. Doğum sancıları başlayınca gece vakti evden çıkar, bebeğini doğurduktan sonra plasenta ve göbek kordonunu ayırmadan bir çöp konteynerine bırakır. Temizlik görevlisi T, bebeği son anda fark ederek hastaneye götürür ve bebek uzun bir tedavi sonrası kurtulur.

\textbf{S'nin cezai sorumluluğu aşağıdakilerden hangisidir?}

\begin{enumerate}[label=\Alph*)]
    \item Altsoya karşı kasten öldürme suçuna teşebbüs (TCK m. 82/1-d, m. 35)
    \item Terk suçu (TCK m. 97)
    \item Taksirle öldürmeye teşebbüs (TCK m. 85, m. 35)
    \item Çocuğa karşı eziyet suçu (TCK m. 96)
    \item Yardım veya bildirim yükümlülüğünün yerine getirilmemesi (TCK m. 98)
\end{enumerate}
\hrule
\vspace{0.3cm}
\textbf{Doğru Cevap: A}
\newline
\textbf{Hukuki Açıklama:} Olayda S, yeni doğmuş bebeğini gece vakti, plasenta ve kordonu ayrılmamış halde bir çöp konteynerine bırakmıştır. Bu eylem, hayat tecrübelerine göre ölüm neticesini muhakkak doğuracak niteliktedir. Bu nedenle S'nin eylemi, basit bir terk (TCK m. 97) olarak değil, doğrudan öldürme kastıyla işlenmiş bir fiil olarak değerlendirilir. Bebek, temizlik görevlisinin tesadüfen fark etmesi gibi failin iradesi dışındaki bir sebeple kurtulduğu için suç, teşebbüs aşamasında kalmıştır. Fiil, kendi çocuğuna (altsoyuna) karşı işlendiği için TCK m. 82/1-d'deki nitelikli hal uygulanır. Terk suçu, kasten öldürme suçu içinde erir (geçitli suç) ve ayrıca cezalandırılmaz.
\textbf{İlgili Kanun Maddeleri:}
\begin{lawbox}
\textbf{TCK Madde 82 - Nitelikli Haller}
\newline
(1) Kasten öldürme suçunun;
a) Tasarlayarak,
b) Canavarca his veya eziyet çektirerek ya da işkence yaparak,
c) Yangın, su baskını, bomba ve benzeri genel tehlike yaratacak araçlarla,
d) Üstsoy veya altsoydan birine ya da eş veya kardeşe karşı,
e) Hamileyken kadına karşı,
f) Çocuğa karşı, kişinin kendisini savunamayacağı bir durumda bulunmasından yararlanarak,
g) Beden veya ruh bakımından kendisini savunamayacak durumda bulunan kişiye karşı,
h) Kamu görevini yapan veya yapmış olan kişiye karşı bu görevi dolayısıyla,
ı) Soykırım veya insanlığa karşı suçlarla bağlantılı olarak,
j) Örgütün faaliyeti çerçevesinde,
k) Terör amacıyla,
l) İşkence suçunun işlenmesi sırasında kasten öldürme gerçekleşirse,
işlenmesi halinde, kişi ağırlaştırılmış müebbet hapis cezası ile cezalandırılır.
\end{lawbox}
\begin{lawbox}
\textbf{TCK Madde 35 - Suça Teşebbüs}
\newline
(1) Kişi, işlemeyi kastettiği bir suçu elverişli hareketlerle doğrudan doğruya icraya başlayıp da elinde olmayan nedenlerle tamamlayamaz ise teşebbüsten dolayı sorumlu tutulur.
\newline
(2) Suça teşebbüs halinde fail, meydana gelen zarar veya tehlikenin ağırlığına göre, ağırlaştırılmış müebbet hapis cezası yerine onüç yıldan yirmi yıla kadar, müebbet hapis cezası yerine dokuz yıldan onbeş yıla kadar hapis cezası ile cezalandırılır. Diğer hallerde verilecek cezanın dörtte birinden dörtte üçüne kadarı indirilir.
\newline
(3) Elverişsiz teşebbüsten dolayı da verilecek ceza üçte ikisi oranında indirilir.
\end{lawbox}
\begin{lawbox}
\textbf{TCK Madde 97 - Terk}
\newline
(1) Bakımakla yükümlü olduğu kişiyi çaresiz durumda bırakan kimse, altı aydan iki yıla kadar hapis cezası ile cezalandırılır.
\newline
(2) Terk sonucunda mağdurun ölümü halinde, bir yıldan beş yıla kadar hapis cezasına hükmolunur.
\end{lawbox}
\begin{lawbox}
\textbf{TCK Madde 98 - Yardım veya bildirim yükümlülüğünün yerine getirilmemesi}
\newline
(1) Kaza, tehlike veya tehdit altında bulunan ve yaşamsal tehlike içinde olan bir kimseye yardım etmekle yükümlü bulunan kişi, durumu adli makamlara, kolluk kuvvetlerine, belediye veya sağlık kuruluşlarına bildirmekle yükümlü olmasına rağmen, gecikmeksizin bu yükümlülüğünü yerine getirmezse, bir yıla kadar hapis veya adlî para cezası verilir.
\newline
(2) Birinci fıkradaki fiillerin ihmali sonucunda kişinin ölümü halinde, altı aydan iki yıla kadar hapis cezasına hükmolunur.
\end{lawbox}


\subsection*{Soru 2}
Cezaevinde infaz koruma baş memuru olan A, koğuşa çıkarılmalarını isteyen B ve C'nin talebini reddeder. Bunun üzerine B ve C, koğuşu yakacaklarını söyler ve yatakları tutuştururlar. A, yangını görmesine ve diğer mahkumların müdahale etme taleplerine rağmen "Ne halleri varsa görsünler" diyerek kapıyı kapatır ve olay yerinden ayrılır. Yangın sonucu B ve C hayatını kaybeder.

\textbf{A'nın, B ve C'nin ölümüyle ilgili cezai sorumluluğu nedir?}

\begin{enumerate}[label=\Alph*)]
    \item İhmali davranışla kasten öldürme suçu (TCK m. 83)
    \item Yangın çıkarmak suretiyle genel güvenliğin kasten tehlikeye sokulması suçuna yardım etme (TCK m. 170, m. 39)
    \item Garantörlük yükümlülüğünü ihlal ederek ihmali davranışla intihara yönlendirme (TCK m. 84)
    \item Taksirle ölüme neden olma (TCK m. 85)
    \item Görevi kötüye kullanma suçu (TCK m. 257)
\end{enumerate}
\vspace{0.5cm}
\hrule
\vspace{0.5cm}
\textbf{Doğru Cevap: C}
\newline
\textbf{Hukuki Açıklama:} İnfaz koruma baş memuru olan A'nın, kanundan (5275 s. Kanun m. 6, 33, 34) kaynaklanan bir garantörlük (koruma ve gözetim) yükümlülüğü vardır. B ve C'nin yangın çıkarma eylemi kendi ölümleri bağlamında bir intihar eylemidir. A, bu eylemi engelleme ve çıkan yangını söndürme imkânı varken, "Ne halleri varsa görsünler" diyerek kayıtsız kalmış ve kapıyı kilitleyerek olay yerini terk etmiştir. Bu ihmali davranışı ile B ve C'nin intiharını kolaylaştırmıştır. A'nın kayıtsızlığı, ölüm neticesini öngördüğünü ve kabullendiğini gösterir (olası kast - TCK m. 21/2). Bu nedenle A, garantörlük yükümlülüğünü ihlal ederek ihmali davranışla intihara yönlendirme (TCK m. 84) suçundan sorumludur.
\vspace{0.5cm}
\textbf{İlgili Kanun Maddeleri:}
\begin{lawbox}
\textbf{TCK Madde 84 - İntihara Yönlendirme}
\newline
(1) Başkasını intihara azmettiren, teşvik eden, başkasının intihar kararını kuvvetlendiren ya da başkasının intiharına herhangi bir şekilde yardım eden kişi, iki yıldan beş yıla kadar hapis cezası ile cezalandırılır.
\newline
(2) İntiharın gerçekleşmesi durumunda, kişi dört yıldan on yıla kadar hapis cezası ile cezalandırılır.
\newline
(3) Yönlendirme ve yardımın etkisiyle intihar teşebbüsünde bulunan kişi etkili bir şekilde tedavi edilmiş olsa bile, fail hakkında birinci fıkra hükmüne göre cezaya hükmolunur.
\newline
(4) İşlediği fiilin anlam ve sonuçlarını algılama yeteneği gelişmemiş olan veya ortadan kaldırılan kişileri intihara sevk edenlerle, cebir veya tehdit kullanmak suretiyle kişileri intihara mecbur edenler, kasten öldürme suçundan sorumlu tutulurlar.
\end{lawbox}
\begin{lawbox}
\textbf{TCK Madde 21 - Kast}
\newline
(1) Suçun oluşması kastın varlığına bağlıdır. Kast, suçun kanuni tanımındaki unsurların bilerek ve istenerek gerçekleştirilmesidir.
\newline
(2) Kişinin, suçun kanuni tanımındaki unsurların gerçekleşebileceğini öngörmesine rağmen, fiili işlemesi halinde olası kast vardır. Bu halde, ağırlaştırılmış müebbet hapis cezasını gerektiren suçlarda müebbet hapis cezasına, müebbet hapis cezasını gerektiren suçlarda yirmi yıldan yirmibeş yıla kadar hapis cezasına hükmolunur; diğer suçlarda ise temel ceza üçte birden yarısına kadar indirilir.
\end{lawbox}

\newpage

\subsection*{Soru 3}
18 yaşındaki A ile 15 yaşındaki B, yaşadıkları duygusal sorunlar nedeniyle birlikte intihar etmeye karar verirler. B'nin babasına ait otomobille nehir kenarına giderler. B aracı nehre sürdüğü esnada, A son anda "Dur!" demesine rağmen araç suya uçar ve A hayatını kaybeder. B ise kurtulur.

\textbf{Hayatta kalan B'nin, A'nın ölümüne ilişkin cezai sorumluluğu nedir?}

\begin{enumerate}[label=\Alph*)]
    \item Taksirle ölüme neden olma (TCK m. 85)
    \item Kasten öldürme suçunun temel hali (TCK m. 81)
    \item Tasarlayarak kasten öldürme suçu (TCK m. 82/1-a)
    \item İntihara yönlendirme suçu (TCK m. 84)
    \item Cezai sorumluluğu yoktur, çünkü ortak intihar kararı vardır.
\end{enumerate}
\vspace{0.5cm}
\hrule
\vspace{0.5cm}
\textbf{Doğru Cevap: B}
\newline
\textbf{Hukuki Açıklama:} Yaşam hakkı, kişinin üzerinde mutlak surette tasarruf edebileceği bir hak değildir. Bu nedenle, bir başkasının hayatına onun rızasıyla son vermek, hukuka uygunluk nedeni sayılmaz ve kasten öldürme suçunu oluşturur. Olayda her ne kadar A ve B'nin ortak intihar kararı olsa da, A'nın ölümüne neden olan aktif fiil (arabayı nehre sürmek) B tarafından gerçekleştirilmiştir. A, son anda "Dur!" diyerek kararından dönme iradesi de göstermiştir. Bu durumda B'nin eylemi, A'nın yaşamını sona erdirmeye yönelik olduğu için kasten öldürme suçunun temel halini (TCK m. 81) oluşturur. B'nin 15 yaşından büyük olması (TCK m. 31/3) ceza miktarında indirim sebebi olacaktır.
\vspace{0.5cm}
\textbf{İlgili Kanun Maddeleri:}
\begin{lawbox}
\textbf{TCK Madde 81 - Kasten Öldürme}
\newline
(1) Bir insanı kasten öldüren kişi, müebbet hapis cezası ile cezalandırılır.
\end{lawbox}
\begin{lawbox}
\textbf{TCK Madde 31 - Yaş Küçüklüğü}
\newline
...
(3) Fiili işlediği sırada onbeş yaşını doldurmuş olup da onsekiz yaşını doldurmamış olan kişiler hakkında suç, ağırlaştırılmış müebbet hapis cezasını gerektirdiği takdirde onsekiz yıldan yirmidört yıla; müebbet hapis cezasını gerektirdiği takdirde oniki yıldan onbeş yıla kadar hapis cezasına hükmolunur. Diğer cezaların üçte biri indirilir ve bu hâlde her fiil için verilecek hapis cezası oniki yıldan fazla olamaz.
\end{lawbox}

\newpage

\subsection*{Soru 4}
A, sosyal medya üzerinden tanıştığı B ile, B 15 yaşını doldurmadan önce birden fazla kez anal yoldan cinsel ilişkiye girer. Bu ilişkiler, B'nin 15 yaşını tamamlamasından sonra da devam eder. Olay, B'nin ailesinin durumu öğrenmesi ve B'nin şikâyetçi olmasıyla adli mercilere intikal eder.

\textbf{A'nın eylemlerine ilişkin uygulanması gereken ceza hukuku kuralları hangi seçenekte doğru verilmiştir?}

\begin{enumerate}[label=\Alph*)]
    \item Sadece zincirleme şekilde çocuğun cinsel istismarı suçundan ceza verilir.
    \item Sadece zincirleme şekilde reşit olmayanla cinsel ilişki suçundan ceza verilir.
    \item Hem çocuğun cinsel istismarı hem de reşit olmayanla cinsel ilişki suçlarından gerçek içtima kurallarınca ayrı ayrı cezalandırılır.
    \item Fiiller bir bütün olarak kabul edilir ve en ağır suçu oluşturan çocuğun cinsel istismarından ceza verilir.
    \item B'nin rızası olduğu için A'ya ceza verilmez.
\end{enumerate}
\vspace{0.5cm}
\hrule
\vspace{0.5cm}
\textbf{Doğru Cevap: C}
\newline
\textbf{Hukuki Açıklama:} Fail A'nın eylemleri, mağdur B'nin yaşına göre iki farklı suç tipini oluşturmaktadır.
\begin{itemize}
    \item \textbf{B 15 yaşını doldurmadan önceki fiiller:} Bu dönemde gerçekleştirilen eylemler, B'nin rızası olsa dahi TCK m. 103 uyarınca zincirleme şekilde (TCK m. 43/1) "çocuğun cinsel istismarı" suçunu oluşturur.
    \item \textbf{B 15 yaşını doldurduktan sonraki fiiller:} Mağdurun yaşının değişmesiyle birlikte, aynı tür eylemler artık farklı bir suç olan TCK m. 104 uyarınca "reşit olmayanla cinsel ilişki" suçunu oluşturur. Bu fiiller de kendi içinde zincirleme suç (TCK m. 43/1) teşkil eder.
\end{itemize}
Yargıtay'ın yerleşik görüşüne göre, bu iki suçun unsurları ve korudukları hukuki menfaatler farklı olduğu için, failin her iki suçtan da gerçek içtima kuralları uygulanması gerekir.
\vspace{0.5cm}
\textbf{İlgili Kanun Maddeleri:}
\begin{lawbox}
\textbf{TCK Madde 103 - Çocuğun Cinsel İstismarı}
\newline
(1) Çocuğu cinsel yönden istismar eden kişi, sekiz yıldan on beş yıla kadar hapis cezası ile cezalandırılır. Cinsel istismarın sarkıntılık düzeyinde kalması hâlinde üç yıldan sekiz yıla kadar hapis cezasına hükmolunur.
\newline
(2) Cinsel istismarın vücuda organ veya sair bir cisim sokulması suretiyle gerçekleştirilmesi durumunda, on altı yıldan aşağı olmamak üzere hapis cezasına hükmolunur.
\newline
(3) Üst soy, ikinci derece dâhil yan söz üçüncü dereceye kadar kan ve kayın hısımları, üvey ebeveyn, veli, vasi, çocuğun bakımı, hizmet veya gözetimiyle görevli kişi, öğretmen veya çocuğun sık sık bulunduğu yerlerin sağlayıcıları tarafından cinsel istismarın işlenmesi halinde, on beş yıldan yirmi yıla kadar hapis cezasına hükmolunur.
\end{lawbox}
\begin{lawbox}
\textbf{TCK Madde 104 - Reşit Olmayanla Cinsel İlişki}
\newline
(1) Cebir, tehdit ve hile olmaksızın, onbeş yaşını bitirmiş olan çocukla cinsel ilişkide bulunan kişi, şikayet üzerine, iki yıldan beş yıla kadar hapis cezası ile cezalandırılır.
\newline
(2) Fail ile mağdur arasında evlilik yaşını doldurmadan önce başlayan ve reşitlik çağına kadar süren bir ilişki bulunması halinde, şikayet halinde dava açılmış olsa bile, mağdurun evlilik yaşına gelmesinden sonra evlenmesi halinde kamu davasının devamına yer olmadığına karar verilir.
\newline
(3) Evlilik sırasında reşit olmayanla cinsel ilişkide bulunmak suç değildir.
\end{lawbox}
\begin{lawbox}
\textbf{TCK Madde 43 - Zincirleme Suç}
\newline
(1) Bir suç işleme kararının icrası kapsamında, değişik zamanlarda bir kişiye karşı aynı suçun birden fazla işlenmesi durumunda, bir cezaya hükmedilir. Ancak bu ceza, dörtte birinden dörtte üçüne kadar artırılır. Bir suçun temel şekli ile daha ağır veya daha az cezayı gerektiren nitelikli şekilleri, aynı suç sayılır. Mağduru belli bir kişi olmayan suçlarda da bu fıkra hükmü uygulanır.
\end{lawbox}

\newpage

\subsection*{Soru 5}
A, B ve C, hızlı tren istasyonundan kablo çalmak üzere yola çıkarlar. A'nın demir kesme makasını kabloya değdirmesiyle üçü de elektrik akımına kapılarak yaralanır. B'nin durumu ağırdır. Olay yerine çağrılan D ile birlikte, öldüğünü düşündükleri B'yi orada bırakırlar. Bir süre sonra geri dönüp yaşadığını gördüklerinde paniğe kapılırlar ve yakalanma korkusuyla hastaneye götürmek yerine B'yi aracın bagajına koyarlar. B yolda hayatını kaybeder.

\textbf{A ve C'nin, B'nin ölümüyle ilgili cezai sorumluluğu nedir?}

\begin{enumerate}[label=\Alph*)]
    \item Taksirle ölüme neden olma (TCK m. 85)
    \item Kasten öldürme (TCK m. 81)
    \item Terk suçu sonucu ölüme neden olma (TCK m. 97)
    \item Önceki tehlikeli fiilleri sebebiyle ihmali davranışla kasten öldürme (TCK m. 83)
    \item Yardım veya bildirim yükümlülüğünün yerine getirilmemesi sonucu ölüme neden olma (TCK m. 98)
\end{enumerate}
\vspace{0.5cm}
\hrule
\vspace{0.5cm}
\textbf{Doğru Cevap: D}
\newline
\textbf{Hukuki Açıklama:} A ve C, hırsızlık amacıyla kablo keserken B'nin ağır yaralanmasına neden olan tehlikeli bir fiil (öncül eylem) gerçekleştirmişlerdir. Bu öncül eylem, onlara yaralanan B'ye yardım etme konusunda hukuki bir yükümlülük (garantörlük/teminat yükümlülüğü) yükler. Başlangıçta B'nin öldüğünü zannederek bıraksalar da, sonradan yaşadığını fark etmelerine rağmen yakalanma korkusuyla onu hastaneye götürmeyerek ölüme terk etmişlerdir. Bu ihmalleri, öncül eylemden kaynaklanan garantörlük yükümlülüğünün ihlalidir. Bu nedenle A ve C, ihmali davranışla kasten öldürme (TCK m. 83) suçundan sorumlu tutulurlar.
\vspace{0.5cm}
\textbf{İlgili Kanun Maddeleri:}
\begin{lawbox}
\textbf{TCK Madde 83 - Kasten Öldürmenin İhmali Davranışla İşlenmesi}
\newline
(1) Kişinin yükümlü olduğu belli bir icrai davranışı gerçekleştirmemesi dolayısıyla meydana gelen ölüm neticesinden sorumlu tutulabilmesi için, bu neticenin oluşumuna sebebiyet veren yükümlülük ihmalinin icrai davranışa eşdeğer olması gerekir.
\newline
(2) İhmali ve icrai davranışın eşdeğer kabul edilebilmesi için, kişinin;
a) Belli bir icrai davranışta bulunmak hususunda kanundan veya sözleşmeden kaynaklanan bir yükümlülüğünün bulunması,
b) Önceden gerçekleştirdiği davranışın başkalarının hayatı ile ilgili olarak tehlikeli bir durum oluşturması,
c) Kanun tarafından yüklenen veya hukuki işlemle üstlenilen koruma, gözetim ve bakım yükümlülüğü,
d) Daha önceki davranışlarıyla başkasının güvenine dayanan özel ilişkilerin varlığı,
Gerekir.
\newline
(3) Belli bir yükümlülüğün ihmali ile ölüme neden olan kişi hakkında, temel ceza olarak, ağırlaştırılmış müebbet hapis cezası yerine yirmi yıldan yirmibeş yıla kadar, müebbet hapis cezası yerine onbeş yıldan yirmi yıla kadar, diğer hallerde ise on yıldan onbeş yıla kadar hapis cezasına hükmolunabileceği gibi, cezada indirim de yapılmayabilir.
\end{lawbox}

\newpage

\subsection*{Soru 6}
19 yaşındaki A, festival alanında 14 yaşını bitirmiş bulunan M'ye arkasından sarılır ve kulağına doğru eğilerek öpmek istediğini söyler. Durumu gören M'nin babası B, A'nın yakasından tutarak onu evinin bahçesine götürür. Ardından ruhsatlı tabancasını alıp gelir ve A'yı yaralar. Raporda A'nın yaralanmasının basit tıbbi müdahale ile giderilemez nitelikte olduğu belirtilir.

\textbf{Baba B'nin A'yı yaralaması eylemine ilişkin doğru hukuki niteleme hangisidir?}

\begin{enumerate}[label=\Alph*)]
    \item Silahla kasten yaralama (TCK m. 86/1, 86/3-e)
    \item Haksız tahrik altında silahla kasten yaralama (TCK m. 86/1, 86/3-e, m. 29)
    \item Kasten öldürmeye teşebbüs (TCK m. 81, m. 35)
    \item Meşru savunma (TCK m. 25)
    \item Neticesi sebebiyle ağırlaşmış yaralama (TCK m. 87)
\end{enumerate}
\vspace{0.5cm}
\hrule
\vspace{0.5cm}
\textbf{Doğru Cevap: B}
\newline
\textbf{Hukuki Açıklama:} B'nin, kızı M'ye yönelik cinsel içerikli bir eyleme (sarkıntılık düzeyinde cinsel istismar - TCK m. 103) tanık olması, kendisinde hiddet ve şiddetli elem meydana getiren haksız bir fiildir. B, bu haksız fiilin etkisi altında A'yı silahla yaralamıştır. Ateş etme mesafesi, hedef alınan bölge (hayati olmayan kol) ve tek el ateş edip devam etmemesi, B'nin kastının öldürmeye değil, yaralamaya yönelik olduğunu gösterir. Yaralanma basit tıbbi müdahale ile giderilemez nitelikte olduğu için fiil, TCK m. 86/1 kapsamındadır. Silahla işlenmesi nitelikli haldir (m. 86/3-e). Kızına yönelik haksız fiilin yarattığı hiddetle hareket ettiği için, B hakkında haksız tahrik hükümlerinin (TCK m. 29) uygulanması ve cezasında indirim yapılması gerekir.
\vspace{0.5cm}
\textbf{İlgili Kanun Maddeleri:}
\begin{lawbox}
\textbf{TCK Madde 86 - Kasten Yaralama}
\newline
(1) Kasten başkasının vücuduna acı veren veya sağlığının ya da algılama yeteneğinin bozulmasına neden olan kişi, bir yıldan üç yıla kadar hapis cezası ile cezalandırılır.
\newline
(2) Kasten yaralama fiilinin basit tıbbi müdahaleyle giderilemeyecek nitelikte olması halinde, iki yıldan beş yıla kadar hapis cezasına hükmolunur.
\newline
(3) Kasten yaralama suçunun;
a) Örgütün faaliyeti çerçevesinde,
b) Terör amacıyla,
c) İşkence niteliğinde,
d) Neticesi itibariyle yaşamsal tehlike oluşturacak biçimde,
e) Silahla,
f) Birden fazla kişiye karşı,
g) Kamu görevini yapan kişiye karşı bu görevi dolayısıyla,
h) Çocuğa karşı,
ı) Etkisi altında bulunduğu kişiye karşı,
i) Hamile olduğu bilinen kadına karşı,
işlenmesi halinde, şikayet aranmaksızın, verilecek ceza yarı oranında artırılır.
\end{lawbox}
\begin{lawbox}
\textbf{TCK Madde 29 - Haksız Tahrik}
\newline
(1) Haksız bir fiilin meydana getirdiği hiddet veya şiddetli elemin etkisi altında suç işleyen kimseye, ağırlaştırılmış müebbet hapis cezası yerine onsekiz yıldan yirmidört yıla ve müebbet hapis cezası yerine oniki yıldan onsekiz yıla kadar hapis cezası verilir. Diğer hallerde verilecek cezanın dörtte birinden dörtte üçüne kadarı indirilir.
\end{lawbox}

\newpage

\subsection*{Soru 7 (Yeni Soru - Olay 2'den Türetilmiştir)}
Cezaevinde B ve C, infaz koruma memuru A'dan kendilerini koğuşa çıkarmasını, bu olmayınca savcı ile görüştürmesini ister. Talepleri reddedilince, "koğuşu yakarız" tehdidinde bulunarak yatakları ateşe verirler. Çıkan yangın sonucu cezaevinde hasar meydana gelir ve B ile C hayatını kaybeder.

\textbf{B ve C'nin, koğuşu yakma eylemi hangi suçları oluşturur?}
\begin{enumerate}[label=\Alph*)]
    \item Sadece intihar (TCK m. 84)
    \item Sadece kasten öldürmeye teşebbüs (TCK m. 81, m. 35)
    \item Kamu malına yakarak zarar verme (TCK m. 152/2-a) ve yangın çıkarmak suretiyle genel güvenliği kasten tehlikeye sokma (TCK m. 170)
    \item Görevi yaptırmamak için direnme (TCK m. 265)
    \item Terör eylemi (TMK m. 1)
\end{enumerate}
\vspace{0.5cm}
\hrule
\vspace{0.5cm}
\textbf{Doğru Cevap: C}
\newline
\textbf{Hukuki Açıklama:} B ve C, yatakları yakarak cezaevine ait olan bir mala (kamu malına) zarar vermişlerdir. Bu eylemin yakarak işlenmesi, suçun nitelikli halini (TCK m. 152/2-a) oluşturur. Ayrıca, kapalı bir ortam olan koğuşta yangın çıkarmak, diğer tutuklu ve hükümlülerin hayatı, sağlığı ve malvarlığı açısından somut bir tehlike yarattığından, aynı zamanda genel güvenliğin kasten tehlikeye sokulması suçunu da (TCK m. 170) oluşturur. B ve C, işledikleri bir fiil ile birden fazla farklı suçun oluşmasına neden olduklarından, farklı neviden fikri içtima (TCK m. 44) hükümleri gereğince en ağır cezayı gerektiren suçtan cezalandırılırlar.
\vspace{0.5cm}
\textbf{İlgili Kanun Maddeleri:}
\begin{lawbox}
\textbf{TCK Madde 152 - Mala Zarar Vermenin Nitelikli Halleri}
\newline
(1) Mala zarar verme suçunun; a) Kamu kurum ve kuruluşlarına ait, kamu hizmetine tahsis edilmiş veya kamunun yararlanmasına ayrılmış yer, bina, tesis veya diğer eşya hakkında, ... işlenmesi hâlinde, fail hakkında bir yıldan dört yıla kadar hapis cezasına hükmolunur.
\newline
(2) Suçun; a) Yakarak, yakıcı veya patlayıcı madde kullanarak, ... işlenmesi hâlinde, verilecek ceza bir katına kadar artırılır.
\end{lawbox}
\begin{lawbox}
\textbf{TCK Madde 170 - Genel Güvenliğin Kasten Tehlikeye Sokulması}
\newline
(1) Kişilerin hayatı, sağlığı veya malvarlığı bakımından tehlikeli olacak biçimde ya da kişilerde korku, kaygı veya panik yaratabilecek tarzda; a) Yangın çıkaran, ... kişi, altı aydan üç yıla kadar hapis cezası ile cezalandırılır.
\end{lawbox}

\newpage

\subsection*{Soru 8 (Yeni Soru - Olay 5'ten Türetilmiştir)}
A, B ve C, Pamukova hızlı tren istasyonundan kablo çalmak üzere anlaşırlar. Gece vakti, demir kesme makası ve plastik eldivenlerle istasyona gelirler. A, demir kesme makasıyla kabloyu kesmeye çalıştığı sırada elektrik akımına kapılırlar ve hırsızlık eylemini tamamlayamadan yaralanırlar.

\textbf{A, B ve C'nin kablo çalmaya yönelik bu eylemlerinin hukuki niteliği nedir?}
\begin{enumerate}[label=\Alph*)]
    \item Tamamlanmış hırsızlık suçu
    \item Gece vakti, kamu hizmetine tahsisli eşya hakkında hırsızlık suçuna teşebbüs
    \item Mala zarar verme suçu
    \item İşyeri dokunulmazlığının ihlali
    \item Hırsızlık suçuna hazırlık hareketi (cezalandırılmaz)
\end{enumerate}
\vspace{0.5cm}
\hrule
\vspace{0.5cm}
\textbf{Doğru Cevap: B}
\newline
\textbf{Hukuki Açıklama:} A, B ve C, çalma kastıyla hareket ederek, demir kesme makasıyla kabloyu kesmeye başlamak suretiyle suçun icra hareketlerine doğrudan başlamışlardır. Ancak elektrik akımına kapılmaları gibi ellerinde olmayan bir nedenle eylemi tamamlayamamışlardır. Bu nedenle fiil, teşebbüs aşamasında kalmıştır (TCK m. 35). Suçun konusunu, kamu hizmetine (hızlı tren) tahsis edilmiş bir eşya (kablo) oluşturmaktadır (TCK m. 142/1-a) ve eylem gece vakti işlenmiştir (TCK m. 143). Dolayısıyla A, B ve C, birlikte suç işleme kararıyla (müşterek faillik - TCK m. 37) nitelikli hırsızlık suçuna teşebbüsten sorumludurlar.
\vspace{0.5cm}
\textbf{İlgili Kanun Maddeleri:}
\begin{lawbox}
\textbf{TCK Madde 141 - Hırsızlık}
\newline
(1) Zilyedinin rızası olmadan başkasına ait taşınır bir malı, kendisine veya başkasına bir yarar sağlamak maksadıyla bulunduğu yerden alan kimseye bir yıldan üç yıla kadar hapis cezası verilir.
\end{lawbox}
\begin{lawbox}
\textbf{TCK Madde 142 - Nitelikli Hırsızlık}
\newline
(1) Hırsızlık suçunun;
a) Kamu kurum ve kuruluşlarında veya ibadete ayrılmış yerlerde bulunan ya da kamu hizmetine veya genel yarara özgülenmiş eşya hakkında,
b) Mesken olarak kullanılan yere girmek suretiyle,
c) Yanlış anahtar, kiloğu veya başka aletlerle kapı, çekmece, sandık, dolap, kasaları ve benzeri kabları açmak suretiyle,
d) Hile ve desise kullanarak,
e) Keçi, köpek, doğan, şahin, kartal, bal değer hayvanlar veya kümes hayvanları hakkında,
f) Sel, yangın, deprem, feyezan veya genel bir afet dolayısıyla boşaltılmış yerlerde,
g) Gece vakti,
h) Birden fazla kişi tarafından birlikte,
ı) Silahla,
j) Örgütün faaliyeti çerçevesinde,
işlenmesi hâlinde, üç yıldan yedi yıla kadar hapis cezasına hükmolunur.
\end{lawbox}
\begin{lawbox}
\textbf{TCK Madde 143 - Suçun Gece Vakti İşlenmesi}
\newline
(1) Hırsızlık suçunun gece vakti işlenmesi halinde, verilecek ceza yarı oranında artırılır.
\end{lawbox}

\newpage

\subsection*{Soru 9 (Yeni Soru - Olay: Bilişim Sistemi)}
Özel bir üniversitede çalışan A, üniversitenin bilgi işlem mühendisi B ve teknisyen C ile anlaşarak, gelir elde etmek amacıyla üniversitenin bilgisayar sunucuları üzerinden sanal para üretimi yapmaya başlar. Bu faaliyet nedeniyle üniversitenin kesintisiz güç kaynağı arızalanır ve otomasyon sisteminde uzun süreli bir yavaşlama meydana gelir.

\textbf{A, B ve C'nin eylemleri bir bütün olarak değerlendirildiğinde, faillerin asli sorumluluğu hangi suçtan kaynaklanır?}
\begin{enumerate}[label=\Alph*)]
    \item Bilişim sistemini engelleme, bozma, verileri yok etme veya değiştirme (TCK m. 244)
    \item Bilişim sistemine girme (TCK m. 243)
    \item Görevi kötüye kullanma (TCK m. 257)
    \item Nitelikli hırsızlık (Elektrik hırsızlığı - TCK m. 142)
    \item Kamu malına zarar verme (TCK m. 152)
\end{enumerate}
\vspace{0.5cm}
\hrule
\vspace{0.5cm}
\textbf{Doğru Cevap: A}
\newline
\textbf{Hukuki Açıklama:} Failler, üniversitenin bilişim sistemine hukuka aykırı olarak girmiş olsalar da (TCK m. 243), eylemleri bununla sınırlı kalmamıştır. Sanal para üretimi faaliyetiyle sistemin işleyişini yavaşlatarak "engellemiş" ve güç kaynağını arızalandırarak "bozmuşlardır". TCK m. 244'te düzenlenen Bilişim Sistemini Engelleme ve Bozma suçu, TCK m. 243'teki Girme suçuna göre daha özel ve ağır bir fiili düzenleyen asli normdur. Bu tür durumlarda, ceza hukukundaki "asli-tali norm" ilişkisi gereğince, failler sadece daha ağır olan asli normdan, yani TCK m. 244'ten sorumlu tutulurlar. Bilişim sistemine girme eylemi, bu suçun içinde erimiş sayılır.
\vspace{0.5cm}
\textbf{İlgili Kanun Maddeleri:}
\begin{lawbox}
\textbf{TCK Madde 244 - Sistemi engelleme, bozma, verileri yok etme veya değiştirme}
\newline
(1) Bir bilişim sisteminin işleyişini engelleyen veya bozan kişi, bir yıldan beş yıla kadar hapis cezası ile cezalandırılır.
\newline
(2) Bir bilişim sistemindeki verileri bozan, yok eden, değiştiren veya erişilmez kılan, sisteme veri yerleştiren, var olan verileri başka bir yere gönderen kişi, altı aydan üç yıla kadar hapis cezası ile cezalandırılır.
\newline
(3) Bu fiillerin bir banka veya kredi kurumuna ya da bir kamu kurum veya kuruluşuna ait bilişim sistemi üzerinde işlenmesi halinde, verilecek ceza yarı oranında artırılır.
\newline
(4) Yukarıdaki fıkralarda tanımlanan fiillerin işlenmesi suretiyle kişinin kendisinin veya başkasının yararına haksız bir çıkar sağlamasının başka bir suç oluşturmaması halinde, iki yıldan altı yıla kadar hapis ve beşbin güne kadar adlî para cezasına hükmolunur.
\newline
(5) Yukarıdaki fıkralarda tanımlanan suçların işlenmesi suretiyle ya da bunların işlenmesi amacıyla ilişki kurulan bilişim sistemlerinin de etkilenmesi halinde, her bir sistem için ayrıca ceza tayin edilir.
\end{lawbox}
\begin{lawbox}
\textbf{TCK Madde 243 - Bilişim Sistemine Girme}
\newline
(1) Hukuka aykırı olarak bir bilişim sistemine giren veya sistemde kalmaya devam eden kişi, bir yıla kadar hapis veya adlî para cezası ile cezalandırılır.
\newline
(2) Yukarıdaki fıkraya göre cezalandırılacak fiillerin bir banka veya kredi kurumuna ya da bir kamu kurum veya kuruluşuna ait bilişim sistemi üzerinde işlenmesi halinde, altı aydan iki yıla kadar hapis cezasına hükmolunur.
\end{lawbox}

\newpage

\subsection*{Soru 10 (Yeni Soru - Olay: Adliye Kalemi)}
Avukat D, suç duyurusunda bulunmak için adliyeye gider. Savcılık kaleminde çalışan memur K, gerçeğe aykırı olarak işlerinin çok yoğun olduğunu, ancak bir miktar para karşılığında dilekçeyi hemen işleme alabileceğini söyler. Avukat D bu talebi reddeder.

\textbf{Memur K'nın cezai sorumluluğu nedir?}
\begin{enumerate}[label=\Alph*)]
    \item İcbar suretiyle irtikap (TCK m. 250/1)
    \item İkna suretiyle irtikap (TCK m. 250/2)
    \item Rüşvet suçuna teşebbüs (TCK m. 252)
    \item Görevi kötüye kullanma (TCK m. 257)
    \item Suç oluşmaz, çünkü D parayı vermemiştir.
\end{enumerate}
\vspace{0.5cm}
\hrule
\vspace{0.5cm}
\textbf{Doğru Cevap: C}
\newline
\textbf{Hukuki Açıklama:} Olayda irtikap suçunun unsurları oluşmamıştır. Çünkü K'nın eylemi, avukat olan D üzerinde onu para vermeye mecbur bırakacak bir "icbar" (zorlama) niteliğinde değildir. Sadece işlerin yoğun olduğunu söylemesi, aldatmaya elverişli bir hileli davranış olmadığından "ikna suretiyle irtikap" da söz konusu olmaz. K'nın eylemi, görevinin ifasıyla ilgili bir işi (dilekçeyi erken kayda alma) yapmak için menfaat temin etmeye yönelik bir taleptir. Bu, TCK m. 252 kapsamında bir rüşvet talebidir. D'nin bu talebi kabul etmemesi nedeniyle eylem, rüşvet anlaşması aşamasına varamamıştır. Bu durum, TCK'da rüşvet suçu için özel olarak düzenlenmiş bir teşebbüs hali oluşturur ve K, rüşvet suçuna teşebbüsten sorumlu tutulur.
\vspace{0.5cm}
\textbf{İlgili Kanun Maddeleri:}
\begin{lawbox}
\textbf{TCK Madde 252 - Rüşvet}
\newline
(1) Görevinin ifasıyla ilgili bir işi yapması veya yapmaması için, doğrudan veya aracılar vasıtasıyla, bir kamu görevlisine veya göstereceği bir başka kişiye menfaat sağlayan kişi, dört yıldan oniki yıla kadar hapis cezası ile cezalandırılır.
\newline
(2) Görevinin ifasıyla ilgili bir işi yapması veya yapmaması için, doğrudan veya aracılar vasıtasıyla, kendisine veya göstereceği bir başka kişiye menfaat sağlayan kamu görevlisi de birinci fıkrada belirtilen ceza ile cezalandırılır.
\newline
(3) Rüşvet konusunda anlaşmaya varılması halinde, suç tamamlanmış gibi cezaya hükmolunur.
\newline
(4) Kamu görevlisinin rüşvet talebinde bulunması ve fakat bunun kişi tarafından kabul edilmemesi ya da kişinin kamu görevlisine menfaat temini konusunda teklif veya vaatte bulunması ve fakat bunun kamu görevlisi tarafından kabul edilmemesi hâllerinde fail hakkında, birinci ve ikinci fıkra hükümlerine göre verilecek cezanın yarısı indirilir.
...
\end{lawbox}

\newpage

\subsection*{Soru 11 (Yeni Soru - Olay: Şoför)}
B'nin azmettirmesiyle şoför F, avukat D'yi önemli bir duruşmaya giderken ıssız bir yolda araçtan indirerek bırakır. Bu eylem sonucunda D duruşmayı kaçırır ve önemli bir vekalet ücretinden mahrum kalır. F, D'yi bırakırken ayrıca belindeki silahı göstererek "Eğer bunu birine anlatırsan kurşunları ayağına boşaltırım" der.

\textbf{F'nin eylemleri bir bütün olarak değerlendirildiğinde cezai sorumluluğu için en doğru niteleme hangisidir?}
\begin{enumerate}[label=\Alph*)]
    \item Sadece kişiyi hürriyetinden yoksun kılma suçu
    \item Silahla tehdit ve kasten yaralamaya teşebbüs suçları
    \item Yağma (Gasp) suçu
    \item Kişiyi hürriyetinden yoksun kılma suçunun silahla işlenen nitelikli hali
    \item Ekonomik kayba neden olan kişiyi hürriyetinden yoksun kılma suçu ve ayrıca silahlı tehdit suçu
\end{enumerate}
\vspace{0.5cm}
\hrule
\vspace{0.5cm}
\textbf{Doğru Cevap: E}
\newline
\textbf{Hukuki Açıklama:} Olayda iki farklı suç işlenmiştir ve bu suçlar arasında gerçek içtima kuralları uygulanmalıdır.
\begin{itemize}
    \item \textbf{Kişiyi Hürriyetinden Yoksun Kılma:} F'nin, D'yi ıssız bir yolda bırakarak bir yere gitme özgürlüğünü engellemesi, TCK m. 109'daki kişiyi hürriyetinden yoksun kılma suçunu oluşturur. Bu suç, D'nin merkezi bir yere ulaşana kadar devam eden mütemadi (kesintisiz) bir suçtur. Eylem sonucunda D'nin davayı kaybederek vekalet ücretinden mahrum kalması, suçun TCK m. 109/3-f'de düzenlenen "mağdurun ekonomik bakımdan önemli bir kaybına neden olması" nitelikli halini oluşturur.
    \item \textbf{Silahlı Tehdit:} F'nin, D'yi bıraktıktan sonra "bunu birine anlatırsan..." diyerek silahla tehdit etmesi, hürriyeti yoksun kılma fiilini kolaylaştırmaya yönelik değil, sonrasındaki şikayeti engellemeye yönelik ayrı bir eylemdir. Bu nedenle, TCK m. 106 kapsamında ayrı bir suç olan tehdit suçunu oluşturur. Tehdidin silahla yapılması, TCK m. 106/2-a uyarınca nitelikli haldir.
\end{itemize}
Bu sebeple F, iki ayrı suçtan (nitelikli kişiyi hürriyetinden yoksun kılma ve nitelikli tehdit) ayrı ayrı cezalandırılacaktır.
\vspace{0.5cm}
\textbf{İlgili Kanun Maddeleri:}
\begin{lawbox}
\textbf{TCK Madde 109 - Kişiyi Hürriyetinden Yoksun Kılma}
\newline
(1) Bir kimseyi hukuka aykırı olarak bir yere gitmek veya bir yerde kalmak hürriyetinden yoksun bırakan kişiye, bir yıldan beş yıla kadar hapis cezası verilir.
...
(2) Mağdurun çocuk olması halinde, ceza altı yıldan oniki yıla kadardır.
\newline
(3) Bu suçun;
a) Örgütün faaliyeti çerçevesinde,
b) Silahla,
c) Birden fazla kişi tarafından,
d) Mağduru tanımamak için maskelenerek,
e) Üniformalı araç kullanarak,
f) Mağdurun ekonomik bakımdan önemli bir kaybına neden olması,
g) Önceden hazırlanmış plan dahilinde,
h) Mala zarar vermek,
ı) Şiddet uygulayarak ya da tehdit ederek,
i) Kamu görevini yapan kişiye karşı bu görevi dolayısıyla,
j) Terör amacıyla,
Halinde, yukarıdaki fıkralara göre verilecek ceza bir kat artırılır.
\end{lawbox}
\begin{lawbox}
\textbf{TCK Madde 106 - Tehdit}
\newline
(1) Bir başkasını, kendisinin veya yakınının hayatına, vücut veya cinsel dokunulmazlığına yönelik bir saldırı gerçekleştireceğinden bahisle tehdit eden kişi, altı aydan iki yıla kadar hapis cezası ile cezalandırılır. Şikayet üzerine soruşturma yapılır.
\newline
(2) Tehdidin;
a) Silahla,
b) Örgütün faaliyeti çerçevesinde,
c) Birden fazla kişi tarafından,
d) Kamu görevini yapan kişiye karşı bu görevi dolayısıyla,
e) Çocuğa karşı,
f) Hamile kadına karşı,
g) Etkisi altında bulunduğu kişiye karşı,
işlenmesi halinde, iki yıldan beş yıla kadar hapis cezasına hükmolunur. Şikayet aranmaz.
\end{lawbox}

\newpage

\subsection*{Soru 12 (Yeni Soru - Olay: Arama Tutanağı)}
C. Savcısı S, avukat D'nin bürosundaki arama işlemine fiilen katılmaz ancak telefonla aradığı kolluk amiri K'ye, arama tutanağının kendisi de katılmış gibi düzenlenmesi talimatını verir. K, emrindeki polis memuru P ile birlikte, S'nin de aramada bulunduğu yönünde gerçeğe aykırı bir tutanak düzenleyip imzalar.

\textbf{Bu olayda Kolluk Amiri K ve Savcı S'nin cezai sorumluluğu sırasıyla nedir?}
\begin{enumerate}[label=\Alph*)]
    \item K: Sorumluluğu yok (Amirin emri) / S: Resmi belgede sahtecilik suçundan fail
    \item K: Görevi kötüye kullanma / S: Görevi kötüye kullanmaya azmettiren
    \item Her ikisi de resmi belgede sahtecilik suçundan müşterek fail
    \item K: Resmi belgede sahtecilik suçundan fail / S: Aynı suça azmettiren
    \item K: Resmi belgede sahtecilik suçuna yardım eden / S: Resmi belgede sahtecilik suçundan fail
\end{enumerate}
\vspace{0.5cm}
\hrule
\vspace{0.5cm}
\textbf{Doğru Cevap: D}
\newline
\textbf{Hukuki Açıklama:}
\begin{itemize}
    \item \textbf{Kolluk Amiri K'nin Sorumluluğu:} Arama tutanağı, kamu görevlisi tarafından görevi gereği düzenlenen bir resmi belgedir. K, savcının aramada bulunmadığını bildiği halde, tutanağa varmış gibi yazarak gerçeğe aykırı bir belge düzenlemiştir. Bu eylem, TCK m. 204/2'de düzenlenen "kamu görevlisinin resmi belgede sahteciliği" (fikri sahtecilik) suçunu oluşturur. Savcının emri, konusu suç teşkil ettiği için hukuka aykırı bir emirdir ve K için bir hukuka uygunluk nedeni (TCK m. 24) oluşturmaz. Bu nedenle K, suçun faili olarak sorumludur.
    \item \textbf{Savcı S'nin Sorumluluğu:} S, suçu bizzat işlememiş ancak suç işleme kararı bulunmayan K'ye bu yönde talimat vererek onda suç işleme kararı oluşturmuştur. Bu durum, TCK m. 38 kapsamında "azmettirme" olarak nitelendirilir. Dolayısıyla S, kamu görevlisinin resmi belgede sahteciliği suçuna azmettiren olarak sorumlu tutulur.
\end{itemize}
\vspace{0.5cm}
\textbf{İlgili Kanun Maddeleri:}
\begin{lawbox}
\textbf{TCK Madde 204 - Resmi Belgede Sahtecilik}
\newline
(1) Bir resmi belgeyi sahte olarak düzenleyen, gerçek bir resmi belgeyi başkalarını aldatacak şekilde değiştiren veya sahte resmi belgeyi kullanan kişi, iki yıldan beş yıla kadar hapis cezası ile cezalandırılır.
\newline
(2) Görevi gereği düzenlemeye yetkili olduğu resmi bir belgeyi sahte olarak düzenleyen, gerçek bir belgeyi başkalarını aldatacak şekilde değiştiren, gerçeğe aykırı olarak belge düzenleyen veya sahte resmi belgeyi kullanan kamu görevlisi üç yıldan sekiz yıla kadar hapis cezası ile cezalandırılır.
\newline
(3) Resmi belge hükmunde olan belgeler hakkında da bu madde hükümleri uygulanır.
\newline
(4) Resmî belgenin aslından çoğaltılması suretiyle sahte belge düzenlenmesi halinde de yukarıdaki fıkralar uygulanır.
\end{lawbox}
\begin{lawbox}
\textbf{TCK Madde 37 - Faillik}
\newline
(1) Suçun kanuni tanımında yer alan fiili birlikte gerçekleştiren kişilerden her biri, fail olarak sorumlu olur.
\newline
(2) Suçun işlenmesinde başkasını aracı olarak kullanan kişi de fail olarak sorumlu tutulur.
\end{lawbox}
\begin{lawbox}
\textbf{TCK Madde 38 - Azmettirme}
\newline
(1) Başkasını suç işlemeye azmettiren kişi, işlenen suçun cezası ile cezalandırılır.
\newline
(2) Azmettiren kişi suçun işlenmesini engellemek veya neticelerin sınırlandırılmasını sağlamak için önemli çaba göstermiş ise, verilecek ceza üçte bire kadar indirilebilir.
\newline
(3) Çocuk veya aklı sır kişinin azmettirene karşı sorumlu olduğu suçları işlemesi halinde, azmettiren kendi fiili olarak sorumlu tutulur.
\end{lawbox}

\end{document}